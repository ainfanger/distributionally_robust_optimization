
\documentclass[12pt]{article}
\usepackage[margin=1in]{geometry} 
\usepackage{amsmath,amsthm,amssymb,amsfonts,mathtools,cancel,enumerate,bm,float}
\usepackage[linesnumbered,ruled,vlined]{algorithm2e}


\newcommand{\N}{\mathbb{N}}
\newcommand{\Z}{\mathbb{Z}}
\newcommand{\R}{\mathbb{R}}
\newcommand{\Q}{\mathbb{Q}}
\newcommand{\ddt}{\frac{d}{dt}}
\newcommand\norm[1]{\left\lVert#1\right\rVert}

\renewcommand{\Pr}{\textbf{Pr}}
\newcommand{\E}{\textbf{E}}
\newcommand{\ind}{\textbf{I}}
\newcommand{\bin}{\textbf{binom}}
\newcommand{\Poisson}{\textbf{Poisson}}
\newcommand{\var}{\textbf{var}}
\newcommand{\cov}{\textbf{cov}}
\newcommand{\prightarrow}{\overset{p}{\to}}
\newcommand{\asrightarrow}{\overset{a.s.}{\to}}
\newcommand{\define}{\overset{\Delta}{=}}
\newcommand{\dist}{\overset{D}{=}}
\newcommand{\cond}{\textbf{cond}}



%Theorems, proofs and solutions.
\theoremstyle{definition}
\newtheorem{theorem}{Theorem}
\theoremstyle{definition}
\newtheorem{lemma}{Lemma}
\newtheorem{remark}{Remark}
\newtheorem{proposition}{Proposition}
\newtheorem{scolium}{Scolium}   %% And a not so common one.
\theoremstyle{definition}
\newtheorem{definition}{Definition}
\newtheorem{corollary}{Corollary}
\newtheorem{example}{Example}
\newenvironment{solution}
  {\begin{proof}[Solution]}
  {\end{proof}}





\setlength{\parindent}{0pt}
\newcommand\numberthis{\addtocounter{equation}{1}\tag{\theequation}}

\newcommand{\overbar}[1]{\mkern 1.5mu\overline{\mkern-1.5mu#1\mkern-1.5mu}\mkern 1.5mu}

\newenvironment{problem}[2][Problem]{\begin{trivlist}
\item[\hskip \labelsep {\bfseries #1}\hskip \labelsep {\bfseries #2.}]}{\end{trivlist}}

\makeatletter
\begin{document}


\title{Notes on Distributionally Robust Optimization for the calculation of Ruin Probabilities}
\author{Alex Infanger}
\maketitle
\emph{All of the following is based on or directly from \cite{JosePaper}.}\\


We work here with the Cramer-Lundberg model for insurance claims. This is a continuous time stochastic process, where the amount of money in the bank $R(t)$ satisfies
\begin{align*}
R(t)=u+ct-\sum_{i=1}^{N_t}X_i.
\end{align*}
Here $u$ is the initial money in the bank, $c$ is the premium rate, and the $X_i$ are claim sizes. They come in at a rate that is a Poisson process, $N_t$. You have to assume the $X_i$ are distributed in some way. Let this distribution have first and second moments $m_1$ and $m_2$ respectively. We can assume $m_1<\infty$ else the insurance company would have no point insuring such a risk. Just for the insurance company to make even, the amount of money coming in has to be the amount of money being paid out on average, which corresponds to the choice
\begin{align*}
c=\nu m_1
\end{align*}
where $\nu$ is the rate at which policies are sold. To make money, the firm adds a safety loading -- the so-called risk premium -- so that,
\begin{align*}
c=(1+\eta)\nu m_1.
\end{align*}
Hence our final model is of the form,
\begin{align*}
R(t)=u+(1+\eta)\nu m_1t-\sum_{i=1}^{N_t}X_i.
\end{align*}
We are interested in the probability of ruin
\begin{align*}
\psi(u,T)\define\Pr\left[\inf_{t\in[0,T]} R(t)\leq 0\right].
\end{align*}
It turns out that it is computationally intractable to solve the above so we instead work with the following Brownian motion approximation,
\begin{align*}
R_B(t)&\define u+(1+\eta)\nu m_1t - (\nu m_1t + \sqrt{\nu m_2}B(t))\\
&=u+\eta \nu m_1t-\sqrt{\nu m_2}B(t).
\end{align*}
Our mission is to robustify this estimate by allowing for some non-trivial movement away from the Brownian motion. For this purpose, we identity the Polish space where the stochastic processes of interest live as the \emph{Skorkhod space},
\begin{align*}
S=D([0,T],\R)
\end{align*}
the space of real valued right-continuous functions with left limits equipped with the $J_1$ metric. The $J_1$ metric is supposed to be like a $\sup$ metric that is robust against time shifts that go to zero. The formal definition of the $J_1$ metric is, if we let $\Lambda$ be the set of strictly increasing functions $\lambda:[0,T]\rightarrow [0,T]$ such that $\lambda\in \Lambda$ implies $\lambda,\lambda^{-1}$ are continuous up to a set of measure zero, and let $e$ be the identity map on $[0,T]$. Then we define
\begin{align*}
d_{J_1}(x_1,x_2)&\define\inf_{\lambda\in \Lambda}\left\{\norm{x_1\circ \lambda -x_2}_{\infty}\lor \norm{\lambda-e}_{\infty}\right\}
\end{align*}
where $\norm{\cdot}_{\infty}$ is the $\sup$ norm and $\lor$ is the max function.\\

Intuitively, this is suppose to capture the sup norm except you're allowed to perturb the function input a litte bit. So it's a bit weaker than the sup norm. Take for example when $T=1$,
\begin{align*}
x_n=(1+n^{-1})\bm{1}_{\left\{[\frac{1}{2}+\frac{1}{n},1]\right\}}, \ \ x = \bm{1}_{\left\{[\frac{1}{2},1]\right\}}.
\end{align*}
Then $\norm{x_n-x}_{\infty}\geq 1$ for all $n$ but $d_{J_1}(x_n,x)\rightarrow 0$ as $n\rightarrow\infty$. To show this notice that,
\begin{align*}
d_{J_1}(x_n,x)&=\inf_{\lambda\in \Lambda}\left\{\norm{x_n\circ \lambda - x}_\infty \lor \norm{\lambda-e}_{\infty}\right\}\\
&\leq \max\left\{\norm{x_n\circ \lambda_n-x}_\infty, \norm{\lambda_n-e}_{\infty}\right\},
\end{align*}
for some feasible choices $\lambda_n$. Let us choose 
\begin{align*}
\lambda_n&=\begin{cases} x & x\in [0,\frac{1}{2}]\\
(1-\frac{2}{n})x + \frac{2}{n}& x\in [\frac{1}{2},1],
\end{cases}
\end{align*}
which is effectively the identity from $[0,\frac{1}{2}]$ and then jumps to $\frac{1}{2}+\frac{1}{n}$ and then is the line which connects $(\frac{1}{2},\frac{1}{2}+\frac{1}{n})$ to $(1,1)$. Notice now that the difference between $x_n\circ \lambda_n$ and $x$ now only sees contribution from the part of the domain $\frac{1}{2}+\frac{1}{n}$, so that,
\begin{align*}
\norm{x_n\circ \lambda_n-x}_{\infty}= \frac{1}{n}.
\end{align*}
On the other hand we have that the identity map and $\lambda$ differ the most at the point $x=\frac{1}{2}$ where they differ by $\frac{1}{n}$, so that we conclude,
\begin{align*}
d_{J_1}(x_1,x_2)&=\inf_{\lambda\in \Lambda}\left\{\norm{x_n\circ \lambda}_\infty \lor \norm{\lambda-e}_{\infty}\right\}\\
&\leq \max\left\{\norm{x_n\circ \lambda_n}_\infty, \norm{\lambda_n-e}_{\infty}\right\}\\
&\leq \frac{1}{n}\rightarrow0.
\end{align*}



\section*{Numerical Example}
In the numerical example from Jose Blanchet and Karthyek Murthy, they use the Paretto distribution for the claim sizes, with $\alpha=2.2$. 


\section*{Questions}
\begin{enumerate}
\item I am worried that I might have implemented that Khosnevisan embedding incorrectly for the following reason: I have to simulate more Brownian motion time steps than Poisson timesteps. This stems from the fact that 
\begin{align*}
\sigma(t)\define\inf\{s: A(s)=m_1t\}.
\end{align*}
But $A(t)=N(t)+S(t)$ is effectively the $\sup$ of a Brownian motion. Because of this it goes like $\sqrt{t}t=o(t)$ such that the above equality (or in the compure it is a $\geq$) exists only for $t$ less than that simulated.

\item How would I implement the large deviations simulation?


\end{enumerate}

\begin{thebibliography}{9}
\bibitem{JosePaper} Jose Blanchet, Karthyek R.A. Murthy. ``Quantifying Distributional Model Risk via Optimal Transport.'' https://arxiv.org/abs/1604.01446.
\end{thebibliography}



\end{document}

